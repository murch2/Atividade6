\documentclass[a4paper,11pt]{article} 
\usepackage{times} 
\usepackage{enumitem}
\usepackage[top=10mm, bottom=15mm, left=20mm, right=20mm]{geometry}
\usepackage{listings}
\lstset{language=Java}
\usepackage[brazil]{babel}
\usepackage[utf8]{inputenc}
\usepackage{amssymb}

\linespread{1.1} 

\newcommand{\sepitem}{\vspace{0.1in}\item} 
\newcommand{\titulo}{\item \textbf{•}}
\begin {document}
\small{
\title{
{\small 
Departamento de Ciência da Computação \hfill IME/USP}\\\vspace{0.1in}
MAC0340 - Laboratório de Engenharia de Software - 2014/S1
}
\vspace{-0.6in}

\author{
	António Augusto Tavares Martins Miranda (7644342) \\
	\textit{amartmiranda@gmail.com}
	\and
	Rodrigo Duarte Louro (7240216)\\
	\textit{digao.louro@gmail.com}
	\vspace{0.1in}
	\and
}

\date{Exercício 6: Avaliação da Adequação de testes}
\maketitle
}
\vspace {-0.4in}
\thispagestyle{empty}

\begin{itemize}

\sepitem \textbf{Ferramentas auxiliares para geração ou análise dos testes}
\setlength{\parindent}{5ex} 

Para a geração dos testes nós não usamos usamos nenhuma ferramenta, ou seja, criamos todos os nossos 
testes manualmente. Para executar os testes usamos a ferramenta \textit{JUnit}.

\sepitem \textbf{Geração dos \textit{test cases} para a análise da cobertura do MC/DC}
\setlength{\parindent}{5ex}

Tal como foi dito no tópico anterior, as malhas de testes, usadas no processo de determinação do grau 
de cobertura do MC/DC, foram elaboradas manualmente, tendo em conta o código fonte alvo e o grau de 
MC/DC que queríamos cobrir. O diretório \textit{Atividade6/src/test/java/malhaDeTestesASerAnalisada} possui 
todas as malhas de teste elaboramos para este protótipo. As malhas de testes que se encontram neste 
diretório são:

\begin{itemize}
\item ExemploClasseUmTeste: esta classe de teste é usada para exercitar a classe ExemploClasseUm 
em relação aos requisitos do MC/DC da mesma. Dentro dele elaboramos dois testes com 100\% de 
cobertura para o MC/DC.
\item ExemploClasseDoisTeste: esta classe de teste é usada para exercitar a classe ExemploClasseDois 
em relação aos requisitos do MC/DC da mesma. Dentro dele elaboramos dois testes com 100\% de 
cobertura para o MC/DC.
\item ExemploClasseTresTeste: esta classe de teste é usada para exercitar a classe ExemploClasseTres 
em relação aos requisitos do MC/DC da mesma. Dentro dele elaboramos dois testes. Um com 100\% de 
cobertura para o MC/DC e outro que nunca vai cobrir 100\% do MC/DC. 
\end{itemize}

\sepitem \textbf{Análise da Adequação da malha de testes}
\setlength{\parindent}{5ex}

Para fazer a avaliação das malhas de teste, estamos aproveitando o \textit{xml} que gerado pelo protótipo do
exercício 4, que contém os requisitos de teste para o cobrimento do MC/DC, e compará-los aos valores 
observados nas variáveis das decisões/condições do código fonte, resultantes da execução da malha de 
testes.
 
Primeiramente extraímos os requisitos de teste para o MC/DC do código fonte, presentes no \textit{xml} gerado 
pelo programa do exercício 4, para uma textit{hashtable} de decisões/condições. Em seguida implantamos diretamente 
no código fonte (do programa a ser testado), operações de captura e armazenamento dos valores das variáveis 
essenciais, particulares a cada decisão/condição do programa e geramos (manualmente) as malhas de teste para 
exercitar o código fonte. Após a execução das malhas de teste é criada uma \textit{hashtable} com as valorações de cada 
uma das variáveis essenciais às decisões/condições do programa e que foram exercitadas pela malha de teste. 
Para computar o grau de cobrimento do MC/DC comparamos as valorações da \textit{hashtable} com os requisitos para o 
cobrimento do MC/DC com a \textit{hashtable} dos valores exercitados pela malha de testes.

\sepitem \textbf{Manual de usuário}
\setlength{\parindent}{5ex}

Antes de executarmos o protótipo temos que ter os seguintes ficheiros:
\begin{itemize}
\item O código fonte que queremos testar;
\item A malha de testes para exercitar o código fonte;
\item o arquivo \textit{xml} do código fonte com os requisitos para o cobrimento do MC/DC.
\end{itemize}
Cabe ao usuário criar os dois primeiros ficheiros. Atenção a malha de testes pode ser gerada automaticamente por 
meio de ferramentas apropriadas.
O arquivo \textit{xml} com os requisitos do MC/DC deve ser gerado (obrigatoriamente) pelo protótipo da atividade 4. 
Dado um código fonte, o protótipo da atividade 4 cria 3 arquivos \textit{xml} cada um com os requisitos para o cobrimento 
dos critérios de todas as condições, todas as decisões e MC/DC. Para mais informações sobre o protótipo da atividade 4
e sobre como usá-lo recomendamos a leitura do relatório do mesmo. Agora que já temos os \textit{xml} com os requisitos do MC/DC
basta colocá-lo no diretório \textit{Atividade6/resources}. Atenção no diretório \textit{Atividade6/resources} só pode existir 
um ficheiro \textit{xml} com os requisitos do MC/DC.
Agora o código fonte a ser testado deve ser instrumentado (manualmente) e só depois adicionado ao diretório \textit{Atividade6/src/main/java/analisar}. 
Seja ClasseExemplo.java,
\begin{lstlisting}
public class ExemploDeClasseDois {
  public static void metodoExemplo (int a, int b) {
    // Codigo executado pelo metodo
    if (a > 0 && b == 0) a = 0;
  }
}
\end{lstlisting}
, o código instrumentado seria,
\begin{lstlisting}
public class ExemploDeClasseDois {
  public static void metodoExemplo (int a, int b) {
    // Instrumentacao
    if (a > 0) {
      HashTable.getInstance()
               .setHashExecutados(``ClasseExemplo.'', ``metodoExemplo.'', 
                                  ``a>0 && b==0'', ``a>0'', true);
    } else {
      HashTable.getInstance()
               .setHashExecutados(``ClasseExemplo.'', ``metodoExemplo.'', 
                                  ``a>0 && b==0'', ``a>0'', false);
    }
    if (b == 0) {
      HashTable.getInstance()
               .setHashExecutados(``ClasseExemplo.'', ``metodoExemplo.'', 
                                  ``a>0 && b==0'', ``b==0'', true);
    } else {
      HashTable.getInstance()
               .setHashExecutados(``ClasseExemplo.'', ``metodoExemplo.'', 
                                  ``a>0 && b==0'', ``b==0'', false);
    }
    // Codigo executado pelo metodo
    if (a > 0 && b == 0) a = 0;    
  }
}
\end{lstlisting}
, ou seja, dada uma decisão, queremos pegar todas as valorações possíveis de cada condição da decisão.

Todos os ficheiros a serem instrumentados têm que incluir \textit{hash.Hastable} e pertencer à \textit{package} 
``analisar'', visto que isso é necessário para a criação da \textit{hash table} dos valores exercitados pela malha de 
testes.

O método \textit{setHashExecutados} recebe como parâmetros a \textit{string} com o nome da classe a ser instrumentada, 
\textit{string} com o nome do método onde se encontra a decisão, \textit{string} da decisão, \textit{string} da condição 
e por fim o valor assumido pela condição. Atenção que as strings que representam classe e o método devem terminar em ponto. 
Exemplo: ``ClasseExemplo.'' e ``metodoExemplo.''.

A malha de testes deve ser colocada no diretório \textit{Atividade6/src/test/java/malhaDeTestesASerAnalisada} com 
\textit{package} ``malhaDeTestesASerAnalisada''. Neste mesmo diretório existe o ficheiro ``TodosOsTestes.java'', onde devemos
acrescentar o nome da malha de testes a ser executada, mais concretamente, no \textit{@SuiteClasses}. Exemplo: 
@SuiteClasses({ClasseExemplo.class})

Agora, finalmente, estamos aptos para rodar o programa. Executando o protótipo como \textit{junit test}, deve ser imprimido 
no \textit{output} o grau de cobertura da malha de teste para cada classe, método e decisão do código fonte.   

\sepitem \textbf{Testes para a validação do protótipo}
\setlength{\parindent}{5ex}

Para a validação do nosso protótipo criamos duas classes de testes. A \textit{HashTest}, visto todas as funcionalidades 
mais importantes do mesmo se encontram implementadas na classe \textit{Hash}. Esta classe contém as implementações da 
montagem das \textit{hashs} como as valores da execução das malhas de teste e os valores necessários para a cobertura do
mcdc, comparação das \textit{hashs} com os valores resultantes da execução das malhas de teste e os valores necessários 
para o cobrimento do MC/DC e por fim a implementação da determinação de cobertura do MC/DC de cada uma da classes
e a sua respectiva impressão. A ``LeituraXMLTest'' é simplesmente para garantir que as condições do MC/DC estão sendo
lidas corretamente do arquivo \textit{xml}.

Na HashTest temos os seguintes testes:
\begin{itemize}
\item setAndGetHashExecutadosTest: este método testa os \textit{getters} e \textit{setters} da \textit{hash table} que contem as 
valorações da execução das malhas de teste;
\item comparaHashTablesTest: este método testa a comparação da \textit{hash table} dos valores exercitados e a \textit{hash table} dos
valores para o cobrimento do MC/DC para a determinação da porcentagem de cobertura do MC/DC. 
\item montaJsonObjectTest: este método testa a transformação da \textit{hash table} num \textit{json object} bem como se as porcentagens 
cobertas pela classe e métodos foram calculadas de maneira correta. 
\end{itemize}

Na ``LeituraXMLTest'' temos o seguinte teste:
\begin{itemize}
\item getRequisitosMCDCTeste: este método testa se a extração dos requisitos do MC/DC está sendo feita corretamente. 
\end{itemize}

\sepitem \textbf{Conclusões}
\setlength{\parindent}{5ex}

A implementação da atividade 6 foi muito tranquila. Acreditamos que isso deve-se grande parte pela atividade 5, onde 
simplesmente pensamos em como implementar a adequação da malha de testes para o MC/DC a partir do protótipo da atividade 4.

As facilidades que tivemos na elaboração do projeto além do conceito da avaliação da adequação da malha de testes
ser simples e da professora não ter exigido uma instrumentação de código automática(muito provavelmente no \textit{bytecode})
foram:

\begin{itemize}
\item reutilização dos requisitos do MC/DC gerados pelo protótipo da atividade 4;
\item leitura e interpretação das informações contidas no \textit{xml} trivial graças à biblioteca \textit{xstream.jar} que usamos 
desde a atividade 4 para gerar o \textit{xml} com os requisitos do MC/DC;
\item instrumentação do código manual e diretamente no código fonte;
\item a linguagem usada para implementar a atividade 6 possui um leque variado de estruturas e recursos (usamos principalmente 
\textit{hash tables} e  \textit{json object}) que nos ajudaram no armazenamento, determinação e impressão do grau de cobertura da malha 
de testes.
\end{itemize}

A única verdadeira dificuldade que tivemos durante a atividade 6 foi o cálculo da porcentagem de cobertura do MC/DC. Esse cálculo 
tinha que ser feito em relação à classe, método e decisão. Isso gerou alguns problemas de duplicação das porcentagens de cobertura
que posteriormente foram resolvidas.

Um ponto negativo no nosso projeto é que ele não foi integrado no protótipo da atividade 4, logo, como dito no manual do usuário 
é necessário usar os dois protótipos para determinarmos o grau de cobertura da malha de testes, ou seja, reduz drasticamente a
usabilidade do mesmo. Além disso como o protótipo é dependente dos requisitos do MC/DC do protótipo da atividade 4, que não trata 
todos os casos do MC/DC, como por exemplo tratamento das máscaras e a negação, então a porcentagem de cobertura do MC/DC apresentada 
por este protótipo não é 100\% coerente com a realidade. Apesar de tudo estamos satisfeitos com os resultados obtidos. 

\end{itemize}

\vfill

\raggedleft
{\sc Junho/2014}

\end{document}
