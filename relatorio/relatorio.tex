\documentclass[a4paper,11pt]{article} 
\usepackage{times} 
\usepackage{enumitem}
\usepackage[top=10mm, bottom=15mm, left=20mm, right=20mm]{geometry}

\usepackage[brazil]{babel}
\usepackage[utf8]{inputenc}

\linespread{1.1} 

\newcommand{\sepitem}{\vspace{0.1in}\item} 
\newcommand{\titulo}{\item \textbf}
\begin {document}
\small{
\title{
{\small 
Departamento de Ciência da Computação \hfill IME/USP}\\\vspace{0.1in}
MAC0340 - Laboratório de Engenharia de Software - 2014/S1
}
\vspace{-0.6in}

\author{
	António Augusto Tavares Martins Miranda (7644342) \\
	\textit{amartmiranda@gmail.com}
	\and
	Rodrigo Duarte Louro (7240216)\\
	\textit{digao.louro@gmail.com}
	\vspace{0.1in}
	\and
}

\date{Exercício 6: Avaliação da Adequação de testes}
\maketitle
}
\vspace {-0.4in}
\thispagestyle{empty}

\begin{itemize}

\sepitem \textbf{Ferramentas auxiliares para geração ou análise dos testes}
\setlength{\parindent}{5ex} 

Para a geração dos testes nós não usamos usamos nenhuma ferramenta, ou seja, criamos todos os nossos 
testes manualmente.

\sepitem \textbf{Geração dos test cases para a análise da cobertura do MC/DC}
\setlength{\parindent}{5ex}

Tal como foi dito no tópico anterior, as malhas de testes, usadas no processo de determinação do grau 
de cobertura do MC/DC, foram elaboradas manualmente, tendo em conta o código fonte alvo e o grau de 
MC/DC que queríamos cobrir. O diretório Atividade6/src/test/java/malhaDeTestesASerAnalisada possui 
todas as malhas de teste elaboramos para este protótipo. As malhas de testes que se encontram neste 
diretório são:

\begin{itemize}
\item ExemploClasseUmTeste: esta classe de teste é usada para exercitar a classe ExemploClasseUm 
em relação aos requisitos do MC/DC da mesma. Dentro dele elaboramos dois testes com 100\% de 
cobertura para o MC/DC.
\item ExemploClasseDoisTeste: esta classe de teste é usada para exercitar a classe ExemploClasseDois 
em relação aos requisitos do MC/DC da mesma. Dentro dele elaboramos dois testes com 100\% de 
cobertura para o MC/DC.
\item ExemploClasseTresTeste: esta classe de teste é usada para exercitar a classe ExemploClasseTres 
em relação aos requisitos do MC/DC da mesma. Dentro dele elaboramos dois testes. Um com 100\% de 
cobertura para o MC/DC e outro que nunca vai cobrir 100\% do MC/DC. 
\end{itemize}

\sepitem \textbf{Análise da Adequação da malha de testes}
\setlength{\parindent}{5ex}

\sepitem \textbf{Manual de usuário}
\setlength{\parindent}{5ex}

\sepitem \textbf{Testes para a validação do protótipo}
\setlength{\parindent}{5ex}

\sepitem \textbf{Conclusões}
\setlength{\parindent}{5ex}

\end{itemize}

\vfill

\raggedleft
{\sc Junho/2014}

\end{document}
