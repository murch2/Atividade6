\documentclass[a4paper,11pt]{article} 
\usepackage{times} 
\usepackage{enumitem}
\usepackage[top=10mm, bottom=15mm, left=20mm, right=20mm]{geometry}
\usepackage{listings}
\lstset{language=Java}
\usepackage[brazil]{babel}
\usepackage[utf8]{inputenc}
\usepackage{amssymb}

\linespread{1.1} 

\newcommand{\sepitem}{\vspace{0.1in}\item} 
\newcommand{\titulo}{\item \textbf{•}}
\begin {document}
\small{
\title{
{\small 
Departamento de Ciência da Computação \hfill IME/USP}\\\vspace{0.1in}
MAC0340 - Laboratório de Engenharia de Software - 2014/S1
}
\vspace{-0.6in}

\author{
	António Augusto Tavares Martins Miranda (7644342) \\
	\textit{amartmiranda@gmail.com}
	\and
	Rodrigo Duarte Louro (7240216)\\
	\textit{digao.louro@gmail.com}
	\vspace{0.1in}
	\and
}

\date{Exercício 6: Avaliação da Adequação de testes}
\maketitle
}
\vspace {-0.4in}
\thispagestyle{empty}

\begin{itemize}

\sepitem \textbf{Ferramentas auxiliares para geração ou análise dos testes}
\setlength{\parindent}{5ex} 

Para a geração dos testes nós não usamos usamos nenhuma ferramenta, ou seja, criamos todos os nossos 
testes manualmente. Para executar os testes usamos a ferramenta JUnit.

\sepitem \textbf{Geração dos test cases para a análise da cobertura do MC/DC}
\setlength{\parindent}{5ex}

Tal como foi dito no tópico anterior, as malhas de testes, usadas no processo de determinação do grau 
de cobertura do MC/DC, foram elaboradas manualmente, tendo em conta o código fonte alvo e o grau de 
MC/DC que queríamos cobrir. O diretório Atividade6/src/test/java/malhaDeTestesASerAnalisada possui 
todas as malhas de teste elaboramos para este protótipo. As malhas de testes que se encontram neste 
diretório são:

\begin{itemize}
\item ExemploClasseUmTeste: esta classe de teste é usada para exercitar a classe ExemploClasseUm 
em relação aos requisitos do MC/DC da mesma. Dentro dele elaboramos dois testes com 100\% de 
cobertura para o MC/DC.
\item ExemploClasseDoisTeste: esta classe de teste é usada para exercitar a classe ExemploClasseDois 
em relação aos requisitos do MC/DC da mesma. Dentro dele elaboramos dois testes com 100\% de 
cobertura para o MC/DC.
\item ExemploClasseTresTeste: esta classe de teste é usada para exercitar a classe ExemploClasseTres 
em relação aos requisitos do MC/DC da mesma. Dentro dele elaboramos dois testes. Um com 100\% de 
cobertura para o MC/DC e outro que nunca vai cobrir 100\% do MC/DC. 
\end{itemize}

\sepitem \textbf{Análise da Adequação da malha de testes}
\setlength{\parindent}{5ex}

Para fazer a avaliação das malhas de teste, estamos aproveitando o xml que gerado pelo protótipo do
exercício 4, que contém os requisitos de teste para o cobrimento do MC/DC, e compará-los aos valores 
observados nas variáveis das decisões/condições do código fonte, resultantes da execução da malha de 
testes.
 
Primeiramente extraímos os requisitos de teste para o MC/DC do código fonte, presentes no xml gerado 
pelo programa do exercício 4, para uma hashtable de decisões/condições. Em seguida implantamos diretamente 
no código fonte (do programa a ser testado), operações de captura e armazenamento dos valores das variáveis 
essenciais, particulares a cada decisão/condição do programa e geramos (manualmente) as malhas de teste para 
exercitar o código fonte. Após a execução das malhas de teste é criada uma hashtable com as valorações de cada 
uma das variáveis essenciais às decisões/condições do programa e que foram exercitadas pela malha de teste. 
Para computar o grau de cobrimentodo mcdc comparamos as valorações da hashtable com os requisitos para o 
cobrimento do mcdc com a hashtable dos valores exercitados pela malha de testes.

\sepitem \textbf{Manual de usuário}
\setlength{\parindent}{5ex}

Antes de executarmos o protótipo temos que ter os seguintes ficheiros:
\begin{itemize}
\item O código fonte que queremos testar;
\item A malha de testes para exercitar o código fonte;
\item o arquivo xml do código fonte com os requisitos para o cobrimento do mcdc.
\end{itemize}
Cabe ao usuário criar os dois primeiros ficheiros. Atenção a malha de testes pode ser gerada automaticamente por 
meio de ferramentas apropriadas.
O arquivo xml com do código fonte com os requisitos do mcdc deve ser pelo protótipo desenvolvido na atividade 4.
Dado um código fonte, o protótipo da atividade 4 cria 3 arquivos XML cada um com os requisitos para o cobrimento 
dos critérios de todas as condições, todas as decisões e mcdc. Para mais informações sobre o protótipo da atividade 4
e sobre como usá-lo recomendamos a leitura do relatório do mesmo. Agora que já temos os xml com os requisitos do mcdc
basta colocá-lo no diretório Atividade6/resources. Atenção no diretório Atividade6/resources só pode existir um ficheiro
xml com os requisitos do mcdc.
Agora o código fonte a ser testado deve ser instrumentado (manualmente) e só depois adicionado ao diretório 
Atividade6/src/main/java/analisar. 
Seja ClasseExemplo.java,
\begin{lstlisting}
public class ExemploDeClasseDois {
  public static void metodoExemplo (int a, int b) {
    // Codigo executado pelo metodo
    if (a > 0 && b == 0) a = 0;
  }
}
\end{lstlisting}
, o código instrumentado seria,
\begin{lstlisting}
public class ExemploDeClasseDois {
  public static void metodoExemplo (int a, int b) {
    // Instrumentacao
    if (a > 0) {
      HashTable.getInstance()
               .setHashExecutados(``ClasseExemplo.'', ``metodoExemplo.'', 
                                  ``a>0 && b==0'', ``a>0'', true);
    } else {
      HashTable.getInstance()
               .setHashExecutados(``ClasseExemplo.'', ``metodoExemplo.'', 
                                  ``a>0 && b==0'', ``a>0'', false);
    }
    if (b == 0) {
      HashTable.getInstance()
               .setHashExecutados(``ClasseExemplo.'', ``metodoExemplo.'', 
                                  ``a>0 && b==0'', ``b==0'', true);
    } else {
      HashTable.getInstance()
               .setHashExecutados(``ClasseExemplo.'', ``metodoExemplo.'', 
                                  ``a>0 && b==0'', ``b==0'', false);
    }
    // Codigo executado pelo metodo
    if (a > 0 && b == 0) a = 0;    
  }
}
\end{lstlisting}
, ou seja, dada uma decisão, queremos pegar todas as valorações possíveis de cada condição da decisão.

Todos os ficheiros a serem instrumentados têm que incluir hash.Hastable e pertencer a package analisar, 
visto que isso é necessário para a criação do hastable dos valores exercitados pela malha de testes.

O método setHashExecutados recebe como parâmetros a string com o nome da classe a ser instrumentada, 
string com o nome do método onde se encontra a decisão, string da decisão, string da condição e por fim 
o valor assumido pela condição. Atenção que as strings que representam classe e o método devem terminar 
em ponto, exemplo ``ClasseExemplo.'' e ``metodoExemplo.''.

A malha de testes deve ser colocada no diretório Atividade6/src/test/java/malhaDeTestesASerAnalisada com 
package malhaDeTestesASerAnalisada. Neste mesmo diretório existe o ficheiro TodosOsTestes.java, onde devemos
acrescentar o nome da malha de testes a ser executada, mais concretamente, no @SuiteClasses. Exemplo: 
@SuiteClasses({ClasseExemplo.class})

Agora, finalmente, estamos aptos para rodar o programa. Executando o protótipo como junit test, deve ser imprimido 
no output o grau de cobertura da malha de teste para cada classe, método e decisão do código fonte.   

\sepitem \textbf{Testes para a validação do protótipo}
\setlength{\parindent}{5ex}

Para a validação do nosso protótipo criamos duas classes de testes. A HashTest, visto todas as funcionalidades 
mais importantes do mesmo se encontram implementadas na classe Hash. Esta classe contém as implementações da 
montagem das hashs como as valores da execução das malhas de teste e os valores necessários para a cobertura do
mcdc, comparação das hashs com os valores resultantes da execução das malhas de teste e os valores necessários 
para o cobrimento do mcdc e por fim a implementação da determinação de cobertura do mcdc de cada uma da classes
e a sua respectiva impressão. A LeituraXMLTest é simplesmente para garantir que as condições do mcdc estão sendo
lidas conrretamente do arquivo xml.

Na HashTest temos os seguintes testes:
\begin{itemize}
\item setAndGetHashExecutadosTest: este método testa os geters e seters da Hashtable que contem as valorações da 
execução das malhas de teste;
\item comparaHashTablesTest: este método testa a comparação da hashtable dos valores exercitados e a hashtable dos
valores para o cobrimento do mcdc para a determinação da porcentagem de cobertura do mcdc. 
\end{itemize}

Na LeituraXMLTest temos o seguinte teste:
\begin{itemize}
\item getRequisitosMCDCTeste: este método testa se a extração dos requisitos do mcdc está sendo feita corretamente. 
\end{itemize}

\sepitem \textbf{Conclusões}
\setlength{\parindent}{5ex}

A implementação da atividade 6 foi muito tranquila. Acreditamos que isso deve-se grande parte pela atividade 5, onde 
simplesmente pensamos em como implementar a adequação da malha de testes para o mcdc a partir do protótipo da atividade 4.

As facilidades que tivemos na elaboração do projeto além do conceito da avaliação da adequação da malha de testes
ser simples e da professora não ter exigido uma instrumentação de código automática(muito provavelmente no bytecode)
foram:

\begin{itemize}
\item reutilização dos requisitos do mcdc gerados pelo protótipo da atividade 4;
\item leitura e interpretação das informações contidas no xml trivial graças à biblioteca xstream.jar que usamos 
desde a atividade 4 para gerar o xml com os requisitos do mcdc;
ividade 4;
\item instrumentação do código manual e diretamente no código fonte;
\item a linguagem usada para implementar a atividade 6 possui um leque variado de estruturas e recursos (usamos principalmente 
hastables e  json object) que nos ajudaram no armazenamento, determinação e impressão do grau de cobertura da malha de testes.
\end{itemize}

A única verdadeira dificuldade que tivemos durante a atividade 6 foi o cálculo da porcentagem de cobertura do mcdc. Esse cálculo 
tinha que ser feito em relação á classe, método e decisão. Isso gerou alguns problemas de duplicação das porcentagens de cobertura
que posteriormente foram resolvidas.

Um ponto negativo no nosso projeto é que ele não foi integrado no protótipo da atividade 4, logo, como dito no manual do usuário 
é necessário usar os dois protótipos para determiarmos o grau de cobertura da malha de testes, ou seja, reduz drasticamente a
usabilidade do mesmo. Além disso como o protótipo é dependente dos requisitos do mcdc do protótipo da atividade 4, que não trata 
todos os casos do mcdc, como por exemplo tratamento das mácaras e a negação, então a porcentagem de cobertura do mcdc apresentada por este 
protótipo não é 100\% coerente com a realidade. Apesar de tudo estamos satisfeitos com os resultados obtidos. 

\end{itemize}

\vfill

\raggedleft
{\sc Junho/2014}

\end{document}
