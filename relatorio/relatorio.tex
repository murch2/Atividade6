\documentclass[a4paper,11pt]{article} 
\usepackage{times} 
\usepackage{enumitem}
\usepackage[top=10mm, bottom=15mm, left=20mm, right=20mm]{geometry}

\usepackage[brazil]{babel}
\usepackage[utf8]{inputenc}

\linespread{1.1} 

\newcommand{\sepitem}{\vspace{0.1in}\item} 
\newcommand{\titulo}{\item \textbf}
\begin {document}
\small{
\title{
{\small 
Departamento de Ciência da Computação \hfill IME/USP}\\\vspace{0.1in}
MAC0340 - Laboratório de Engenharia de Software - 2014/S1
}
\vspace{-0.6in}

\author{
	António Augusto Tavares Martins Miranda (7644342) \\
	\textit{amartmiranda@gmail.com}
	\and
	Rodrigo Duarte Louro (7240216)\\
	\textit{digao.louro@gmail.com}
	\vspace{0.1in}
	\and
}

\date{Exercício 6: Avaliação da Adequação de testes}
\maketitle
}
\vspace {-0.4in}
\thispagestyle{empty}

\begin{itemize}

\sepitem \textbf{Ferramentas auxiliares para geração ou análise dos testes}
\setlength{\parindent}{5ex} 

Para a geração dos testes nós não usamos usamos nenhuma ferramenta, ou seja, criamos todos os nossos 
testes manualmente. Para executar os testes usamos a ferramenta JUnit.

\sepitem \textbf{Geração dos test cases para a análise da cobertura do MC/DC}
\setlength{\parindent}{5ex}

Tal como foi dito no tópico anterior, as malhas de testes, usadas no processo de determinação do grau 
de cobertura do MC/DC, foram elaboradas manualmente, tendo em conta o código fonte alvo e o grau de 
MC/DC que queríamos cobrir. O diretório Atividade6/src/test/java/malhaDeTestesASerAnalisada possui 
todas as malhas de teste elaboramos para este protótipo. As malhas de testes que se encontram neste 
diretório são:

\begin{itemize}
\item ExemploClasseUmTeste: esta classe de teste é usada para exercitar a classe ExemploClasseUm 
em relação aos requisitos do MC/DC da mesma. Dentro dele elaboramos dois testes com 100\% de 
cobertura para o MC/DC.
\item ExemploClasseDoisTeste: esta classe de teste é usada para exercitar a classe ExemploClasseDois 
em relação aos requisitos do MC/DC da mesma. Dentro dele elaboramos dois testes com 100\% de 
cobertura para o MC/DC.
\item ExemploClasseTresTeste: esta classe de teste é usada para exercitar a classe ExemploClasseTres 
em relação aos requisitos do MC/DC da mesma. Dentro dele elaboramos dois testes. Um com 100\% de 
cobertura para o MC/DC e outro que nunca vai cobrir 100\% do MC/DC. 
\end{itemize}

\sepitem \textbf{Análise da Adequação da malha de testes}
\setlength{\parindent}{5ex}

Para fazer a avaliação das malhas de teste, estamos aproveitando o xml que gerado pelo protótipo do
exercício 4, que contém os requisitos de teste para o cobrimento do MC/DC, e compará-los aos valores 
observados nas variáveis das decisões/condições do código fonte, resultantes da execução da malha de 
testes.
 
Primeiramente extraímos os requisitos de teste para o MC/DC do código fonte, presentes no xml gerado 
pelo programa do exercício 4, para uma hashtable de decisões/condiçõe. Em seguida implantamos diretamente 
no código fonte (do programa a ser testado), operações de captura e armazenamento dos valores das variáveis 
essenciais, particulares a cada decisão/condição do programa e geramos (manualmente) as malhas de teste para 
exercitar o código fonte. Após a execução das malhas de teste é criada uma hashtable com as valorações de cada 
uma das variáveis essenciais às decisões/condições do programa e que foram exercitadas pela malha de teste. 
Para computar o grau de cobrimentodo mcdc comparamos as valorações da hashtable com os requisitos para o 
cobrimento do mcdc com a hashtable dos valores exercítados pela malha de testes.

\sepitem \textbf{Manual de usuário}
\setlength{\parindent}{5ex}

...

\sepitem \textbf{Testes para a validação do protótipo}
\setlength{\parindent}{5ex}

Para a validação do nosso protótipo criamos duas classes de testes. A HashTest, visto todas as funcionalidades 
mais importantes do mesmo se encontram implementadas na classe Hash. Esta classe contém as implementações da 
montagem das hashs como as valores da execução das malhas de teste e os valores necessários para a cobertura do
mcdc, comparação das hashs com os valores resultantes da execução das malhas de teste e os valores necessários 
para o cobrimento do mcdc e por fim a implementação da determinação de cobertura do mcdc de cada uma da classes
e a sua respectiva impressão. A LeituraXMLTest é simplesmente para garantir que as condições do mcdc estão sendo
lidas conrretamente do arquivo xml.

Na HashTest temos os seguintes testes:
\begin{itemize}
\item setAndGetHashExecutadosTest: este método testa os geters e seters da Hashtable que contem as valorações da 
execução das malhas de teste;
\item comparaHashTablesTest: este método testa a comparação da hashtable dos valores exercitados e a hashtable dos
valores para o cobrimento do mcdc para a determinação da porcentagem de cobertura do mcdc. 
\end{itemize}

Na LeituraXMLTest temos o seguinte teste:
\begin{itemize}
\item getRequisitosMCDCTeste: este método testa se a extração dos requisitos do mcdc está sendo feita corretamente. 
\end{itemize}

\sepitem \textbf{Conclusões}
\setlength{\parindent}{5ex}

...

\end{itemize}

\vfill

\raggedleft
{\sc Junho/2014}

\end{document}
